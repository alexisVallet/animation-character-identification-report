\subsection{Classification}
\subsubsection{Tree-walk kernels}
Our initial idea for classifying the animation characters is to consider the region adjacency graph on the segments of the image. The intuition behind is that although changes in posture, scale change the overall image significantly, the arrangement of segments in the image relative to each other should roughly stay the same. 

Given some animation image $I$, some segmentation $S = \{S_1, ..., S_q\}$ of $I$ and a $4$-connected or $8$-connected notion of adjacency between pixels. The region adjacency graph $G$ of $I$ is defined as the graph where vertices are segments and there is an edge between segments $S_i$ and $S_j$ if and only if there is a pair of pixels $(p_i, p_j) \in S_i \times S_j$ such that $p_i$ and $p_j$ are adjacent.

\subsubsection{Segmentation graph spectral classification methods}

\subsubsection{Segment matching method}
