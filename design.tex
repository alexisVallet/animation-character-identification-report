\section{Design of the method}

\begin{figure}[htb!]
\centerline{
\includegraphics[height=1.2\textwidth,angle=270,clip=true,trim=0 0 6cm 0]{images/visionSystemDiagram.pdf}
}
\caption{Diagram depicting how preprocessing, segmentation and classification interact.}
\end{figure}

The method takes the shape of computer vision system with separate phases of preprocessing, segmentation and classification.

\begin{itemize}
\item The preprocessing step should operate transformations on the original image so segmentation can be applied efficiently and accurately. This includes filtering, color space transformations, histogram manipulations and geometric transformations. This results into a new preprocessed image.
\item The segmentation method aims to automatically isolate parts of the character image as faithfully to human perception as possible. An ideal segmentation would for instance segment separately hair, face, clothes, hands and feet as separate components.
The end result of the segmentation is a partition of the set of pixels in the image.
\item The goal of classification is to extract features of interest from the segmentation, and use these features to predict the character identity from the training information. The end result is the predicted name of the character.
\end{itemize}

Although the global structure of the method described above did not change during design, many techniques were experimented with for each of the steps. In this section each technique will be described.

\section{Conventions}
\label{sec:conventions}

This section establishes the mathematical conventions and definitions used in this document. They are provided to make this document self-contained, please see "Spectral Graph Theory" by Fan R. K. Chung \cite{chung1997spectral} (from which these conventions are borrowed) for a more thorough treatment of spectral graph theory.

\subsection{Spectral graph theory}
In this document, a(n undirected) \emph{graph} $G = (V, E)$ consists of a \emph{vertex set} $V$ and an \emph{edge set} $E \subseteq \{\{u, v\} | u \in V \text{ and } v \in V\}$. We say that vertices $u$ and $v$ are \emph{adjacent}, denoted $u \sim v$, if and only if $\{u,v\}$ is an edge. A \emph{weighted graph} consist of a graph $G = (V, E)$ with an associated \emph{weight function} $w : V \times V \rightarrow \mathbb{R}^+ \cup \{0\}$ where $w$ is symmetric and there is no edge $\{u,v\}$ if and only if $w(u,v) = w(v,u) = 0$ . A graph is called \emph{simple} if it has no loops (e.g. no edge $\{u, v\}$ where $u = v$). A graph is called \emph{finite} if its vertex set is finite. Unless indicated otherwise, we assume graphs to be undirected simple finite weighted graphs. We note that unweighted graphs can be seen as weighted graphs with weight function $w$ defined by:

\[
w(u,v) = \begin{cases}
1 & \text{if }u \sim v \\
0 & \text{otherwise}
\end{cases}
\]

Common graphs used in this document include:
\begin{itemize}
\item the \emph{$K$-nearest-neighbor graphs}, which given feature $f  : V \rightarrow \mathbb{R}^q$ for $q \in \mathbb{N} - \{0\}$ connects a pixel to its $K$ nearest neighbors by euclid distance in feature space. Since the $K$-nearest-neighbor membership relation is not symmetric, we distinguish between $2$ cases \cite{von2007tutorial}:
\begin{itemize}
\item The graph where there is an edge between $u$ and $v$ if  and only if $v$ is one of $u$'s $K$ nearest neighbors \emph{or} $u$ is one of $v$'s $K$ nearest neighbors, which we simply call \emph{$K$-nearest-neighbor graph}.
\item The graph where there is an edge between $u$ and $v$ if and only if $v$ is one of $u$'s $K$ nearest neighbors \emph{and} $u$ is one of $v$'s $K$ nearest neighbors, which we simply call \emph{mutual $K$-nearest-neighbor graph}.
\end{itemize}
\item the \emph{complete graph} in which all pairs of distinct vertices have an edge between them.
\end{itemize}

Let $G = (V,E)$ be a graph. For notational convenience, and since we assume graphs to be finite, we will interchangeably use the vertex and its index in the $\{1, ..., |V|\}$ set. We define the \emph{adjacency matrix} $A \in \mathbb{R}^{n \times n}$ of $G$ where $n = |V|$ by $A_{uv} = w(u,v)$, and the \emph{degree} $d_u$ of vertex $u$ by $d_u = \sum_{u \sim v} w(u,v)$. Let $D = diag(d_1, ..., d_n)$ be the \emph{degree matrix} of $G$.

We define the \emph{combinatorial Laplacian} $L \in \mathbb{R}^{n \times n}$ of $G$ by the following:

\[
L = D - A \quad \Leftrightarrow \quad L_{uv} = \begin{cases}
d_u & \text{if }u = v \\
-w(u,v) & \text{if }u \sim v \\
0 & \text{otherwise}
\end{cases}
\]

the \emph{normalized Laplacian}\footnotemark[1] $\mathcal{L} \in \mathbb{R}^{n \times n}$ by:

\footnotetext[1]{
Both the combinatorial and normalized Laplacian are commonly refered to as "the Laplacian of $G$" in the literature. To avoid confusion, in this document we will use these names as they were defined in \cite{von2007tutorial}.
}

\[
\mathcal{L} = D^{-\frac{1}{2}}LD^{-\frac{1}{2}} = I - D^{-\frac{1}{2}}AD^{-\frac{1}{2}} \quad \Leftrightarrow \quad \mathcal{L}_{uv} = \begin{cases}
1 & \text{if } u = v \text{ and } d_u \neq 0\\
-\frac{w(u,v)}{\sqrt{d_ud_v}} & \text{if } u \sim v\\
0 & \text{otherwise}
\end{cases}
\]
where
\[
D^{-\frac{1}{2}}_{u,v} = \begin{cases}
\frac{1}{\sqrt{d_u}} & \text{if }u = v\text{ and }d_u \neq 0\\
0 & \text{otherwise}
\end{cases}
\]

and the \emph{random walk Laplacian}\footnotemark[2] $L_{rw} \in \mathbb{R}^{n \times n}$ of $G$ by:

\footnotetext[2]{
The name comes from the fact that this matrix is closely related to the transition probability matrix of a random walk defined on $G$ \cite{von2007tutorial}.
}

\[
L_{rw} = D^{-1}L = I - D^{-1}A \quad \Leftrightarrow \quad (L_{rw})_{uv} = \begin{cases}
1 & \text{if }u = v \text{ and }d_u \neq 0 \\
-\frac{w(u,v)}{d_u} & \text{if }u \sim v \\
0 & \text{otherwise}
\end{cases}
\]

and $D^{-1}$ is defined similarly to $D^{-\frac{1}{2}}$.

\subsection{Image processing}
In this document, an \emph{image} $I$ is a function $I : \{1, ..., h\} \times \{1, ..., w\} \rightarrow \{0, ..., 2^d - 1\}^c$ where $h$ is the \emph{height} of the image, $w$ its \emph{width}, $d$ its \emph{depth} and $c$ its \emph{number of channels}. When $c = 3$, $I$ is called a \emph{color image}.

Let $I$ be an image, and $G = (V,E)$ be a graph where $V = dom(I)$ is the set of pixel coordinates of $I$. A \emph{segmentation} $S = \{S_1, ..., S_m\}$ is a partition of $V$ where each $S_i$ is a connected component in $G$. Common graphs considered for segmentation are:

\begin{itemize}
\item the \emph{$4$-connected graph}, which connects pixels directly under, over, left and right of each pixel.
\item the \emph{$8$-connected graph}, which is the $4$-connected graph with additional diagonal edges.
\item the $K$-nearest-neighbors graphs for some pixel features - usually a mix of color and coordinates information.
\end{itemize}

\subsection{Preprocessing}
\subsubsection{Kuwahara filter}
\begin{figure}[htb!]
\centering
\begin{subfigure}{.24\textwidth}
\includegraphics[width=\textwidth]{images/miku_d.png}
\caption{Before filtering.}
\end{subfigure}
\begin{subfigure}{.24\textwidth}
\includegraphics[width=\textwidth]{images/miku_d_filtered_smallh.png}
\caption{Small window.}
\end{subfigure}
\begin{subfigure}{.24\textwidth}
\includegraphics[width=\textwidth]{images/miku_d_filtered.png}
\caption{"good" window.}
\end{subfigure}
\begin{subfigure}{.24\textwidth}
\includegraphics[width=\textwidth]{images/miku_d_filtered_largeh.png}
\caption{Large window.}
\end{subfigure}
\caption{Results of Kuwahara filtering with varying window size. The small window causes outlines to be accentuated, and the large window removed the tie, ribbon and legs.}
\label{fig:kuwaharaExample}
\end{figure}

One of the early issues encountered when attempting to segment animation character images was the presence of black outlines, which usually get segmented into many large components, but do not contain any information relevant to identification. We identified the Kuwahara filter \cite{kuwahara1976processing} as a way to not only remove the black outlines, but also remove other unnecessary details and make area of color more homogeneous and amenable to segmentation.

\paragraph{The method} Consider a grayscale image $I$ (we'll see later how it generalizes to color images). For each point $(x,y)$ on $I$, we consider a square window of size $2h + 1$ centered on it. We then consider $4$ overlapping square regions in this window as shown in \autoref{fig:kuwaharaQuadrants}. For each such region $Q_i$ for $i \in \{1, ..., 4\}$ we compute the arithmetic mean $m_i(x,y)$ and standard deviation $\sigma_i(x,y)$ of pixel values inside $Q_i$. The pixel value at $(x,y)$ in filtered image $J$ is then defined as the arithmetic mean $m_i(x,y)$ corresponding to the smallest standard deviation $\sigma_i(x,y) = \min_{j \in \{1, ..., 4\}}\sigma_j(x,y)$. When dealing with borders of the image, we simply do not consider the missing pixels while computing the means and variances.

\begin{figure}[htb!]
\centering
\includegraphics[width=0.3\textwidth]{images/kuwahara_quadrants.jpg}
\caption{Window used by the Kuwahara filter with half-size $h = 2$ and square regions $a$, $b$, $c$ and $d$.}
\label{fig:kuwaharaQuadrants}
\end{figure}

Generalizing to color images, we still consider the arithmetic means of pixel colors, but this time we consider the brightness of the pixels (e.g. the V in HSV color space) for standard deviation.

\paragraph{Theoretical justification} The algorithm was originally designed to improve segmentation in medical image processing. For the case of removing outlines, if one considers a window size "large enough" that outlines never fills most of it, then it will introduce a large standard deviation in the square region it is contained in - meaning it will be removed in the filtered image. It should be noted that if the window is not large enough, then outlines can actually be thickened by the Kuwahara filter, and if it is too large then relevant information may be lost (\autoref{fig:kuwaharaExample}). We found however that it is easy do empirically determine a window size in practice.

\paragraph{Implementation} A naive implementation of the filter recomputes all means and variances for each pixel, runs in $O(nh^2)$ time where $n$ is the number of pixels in the image, and $h$ is the window half size.

\subsubsection{Color space selection}

As the goal is to obtain a segmentation as close to human perception as possible, the $L^*a^*b^*$ color space has been found to be most suitable. The $HSV$ or $HSL$ color spaces are also useful to extract hue information, (the $H$ in either color space) which we use for histogram equalization (\autoref{sec:histEq}) and as post-processing to segmentation (\autoref{sec:hueMerging}).

Although using $L^*a^*b^*$ color space gives good segmentations, it may not be ideal for classification as $2$ different characters may share predominantly the same color palette. A possible extension of this would be to use training data to determine an abstract (possibly high-dimensional) color space where members of the same class are close and members of different classes are far away, using - for instance - a (semi) supervised embedding method similar to the one used in \cite{urahama2007semi}.

\subsubsection{Color histogram equalization by hue}
\label{sec:histEq}

Some characters are represented predominantly by a few colors, which makes segmentation more difficult. Equalizing the histogram by hue allows segmentation to distinguish between more closely related colors, and therefore distinguish some colors which would be otherwise close in color space. The drawback beinng that irrelevant differences in color space are also accentuated, so histogram equalization was eventually removed from preprocessing as it had an overall negative impact on segmentation performance.

\subsection{Segmentation}
\subsubsection{Felzenszwalb's method}

\subsubsection{Spectral segmentation algorithms}

\subsubsection{Felzenszwalb's method with hue-based segment fusion}


\subsection{Classification}
\subsubsection{Tree-walk kernel}
Our initial idea for classifying the animation characters is to consider the region adjacency graph on the segments of the image. The intuition is that although changes in posture, scale change the overall image significantly, the arrangement of segments in the image relative to each other should roughly stay the same. Classifying images by their region adjacency graph has been studied by Z. Harchaoui and F. Bach in \cite{harchaoui2007image}, in which they a introduce a number of kernels to compute similarity between such graphs. We chose to study and implement the tree-walk kernel presented in this paper.

\paragraph{The method} Let $I$ be a color image, $S = \{S_1, ..., S_q\}$ a segmentation of $I$ and a $4$-connected notion of adjacency between pixels. The region adjacency graph $G$ of $I$ is defined as the graph where vertices are segments and there is an edge between segments $S_i$ and $S_j$ if and only if there is a pair of adjacent pixels $(p_i, p_j) \in S_i \times S_j$. As the $4$-connected graphs are always planar, and the region adjacency graph is a minor of the $4$-connected graph, the region adjacency graph is planar by Wagner's theorem. It should be noted however that the $8$-connected graphs are not planar beyond a grid of size 5 by 5 vertices by Euler's formula on planar graphs, and so using an $8$-connected notion of adjacency could be inappropriate for this method.

Let $G$ and $H$ be region adjacency graphs with labeling functions $l_g : V(G) \rightarrow L$ and $l_h : V(H) \rightarrow L$, where $L$ is a segment label set. The method considers a kernel function $k : L \times L \rightarrow \mathbb{R}^+ \cup \{0\}$ which measures similarity between segment labels. The authors first define the $p$-th order walk kernel $k_{\mathcal{W}}^p(G,H)$ between $G$ and $H$ as:

\[
k_{\mathcal{W}}^p(G,H) = \sum_{\substack{(r_1, ...., r_p) \in \mathcal{W}_G^p\\ (s_1, ...., s_p) \in \mathcal{W}_G^p}} \prod_{i=1}^p k(l_G(r_i), l_H(s_i))
\]

where $\mathcal{W}_G^p$ (resp $\mathcal{W}_H^p$) denotes the set of walks of length at most $p$ in $G$ (resp $H$).

\subsubsection{Segmentation graph spectral classification methods}
The tree walk kernels method performed poorly in part because it relies overly on the region adjacency graph of the image, whose structure changes significantly with image variations. In an effort to match graph structure more loosely, we considered the results from spectral graph theory which attempt to analyze graphs through the eigenvalues and eigenvectors of its Laplacian. Works by Wilson, Zhu, Hancock and Luo showed that such spectral methods could be used for graph pattern matching, including image classification through shock graphs \cite{wilson2005pattern}\cite{wilson2008study}. Furthermore, works in graph cut algorithms and application to clustering \cite{ng2002spectral} and segmentation\cite{shi2000normalized}\cite{meila2001random} showed that the eigenvectors of the Laplacian corresponding to smaller eigenvalue encode in a sense the global structure of the graph, while ignoring smaller details. We hoped this would allow the method to be robust to variations in animation character images.

\paragraph{The method} For each segmentation $S$, we consider $m$ features $(f_i : S \Rightarrow \mathbb{R}_i^q)_{1 \leq i \leq m}$. For instance, we used average $L*a*b*$ color, gravity center and area of the segment as features. For each feature $f_i$, we compute the $K$-nearest neighbor graph $G_i$ on $S$ with edges weighted by the Gaussian kernel $w(S_u,S_v) = e^{-\frac{||f_i(S_u) - f_i(S_v)||^2}{\sigma_i^2}}$, and its corresponding combinatorial Laplacian  $L_i$ (see \autoref{sec:laplacians} for definitions).




\subsubsection{Segment matching method}

