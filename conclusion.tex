\addcontentsline{toc}{section}{Conclusion}
\section*{Conclusion}

My work as a research student in Kyushu University led to significant development in animation character identification. Indeed, while there wasn't much state of the art prior to this work, our current method achieves a $59$\% recognition rate on a dataset with $12$ characters, which is - at the very least - much better than random, which would give a roughly $8$\% recognition rate. Furthermore, we show experimentally that this number scales well with the size of the dataset. Although our method still suffers from issue related to characters sharing similar color schemes, we provide in this document the outline of a method which could solve these issues and we give general guidelines for further extensions. My work led to the acceptance of a paper at the Kyushu JCEEE conference, which I successfully presented at the international sessions at the end of August in Kumamoto.

During my work here, I learnt a lot through the study of numerous research papers, most notably in the fields of image processing, machine learning and more precisely applications of spectral graph theory in dimensionality reduction, clustering, (semi) supervised classification and graph pattern matching, the principles of kernel methods and their many applications.

I can therefore say that my work at Kyushu University has been very fruitful, and I plan to come back for a Ph.D. under the supervision of professor Sakamoto to extend our work on animation images analysis.