\section{A method for identifying animation characters from color images}
\subsection{Subject}
The subject of my work here is to design a computer algorithm able to automatically identify characters from color images. It should take the shape of a supervised or semi-supervised classification method, meaning it should be able to identify an image by comparing it against a training set. Said training set may consist of labeled images - images which are known to depict a certain character - as well as unlabeled images in the case of semi-supervised classification.

\begin{figure}[htb!]
% TODO: data flow diagram for the method
\caption{Data flow diagram depicting how the method should work.}
\label{fig:methodDiagram}
\end{figure}

\subsection{Position in the laboratory}
I worked in the Sakamoto laboratory as a research student.
\subsection{Previous studies}
Previous work in the laboratory by graduate student Yuki Nakagawa on the animation character identification problem identified a segmentation method\cite{felzenszwal04} and a classification\cite{harchaoui07} method for the problem.
\subsection{Objectives}
The object
\subsection{Planning}
\subsection{Design of the method}
